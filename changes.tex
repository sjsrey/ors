\documentclass{article}
%DIF LATEXDIFF DIFFERENCE FILE
%DIF DEL submitted.tex     Mon Apr 14 09:26:07 2014
%DIF ADD ors_address.tex   Mon Apr 14 09:17:12 2014
%DIF PREAMBLE EXTENSION ADDED BY LATEXDIFF
%DIF UNDERLINE PREAMBLE %DIF PREAMBLE
\RequirePackage[normalem]{ulem} %DIF PREAMBLE
\RequirePackage{color}\definecolor{RED}{rgb}{1,0,0}\definecolor{BLUE}{rgb}{0,0,1} %DIF PREAMBLE
\providecommand{\DIFadd}[1]{{\protect\color{blue}\uwave{#1}}} %DIF PREAMBLE
\providecommand{\DIFdel}[1]{{\protect\color{red}\sout{#1}}}                      %DIF PREAMBLE
%DIF SAFE PREAMBLE %DIF PREAMBLE
\providecommand{\DIFaddbegin}{} %DIF PREAMBLE
\providecommand{\DIFaddend}{} %DIF PREAMBLE
\providecommand{\DIFdelbegin}{} %DIF PREAMBLE
\providecommand{\DIFdelend}{} %DIF PREAMBLE
%DIF FLOATSAFE PREAMBLE %DIF PREAMBLE
\providecommand{\DIFaddFL}[1]{\DIFadd{#1}} %DIF PREAMBLE
\providecommand{\DIFdelFL}[1]{\DIFdel{#1}} %DIF PREAMBLE
\providecommand{\DIFaddbeginFL}{} %DIF PREAMBLE
\providecommand{\DIFaddendFL}{} %DIF PREAMBLE
\providecommand{\DIFdelbeginFL}{} %DIF PREAMBLE
\providecommand{\DIFdelendFL}{} %DIF PREAMBLE
%DIF END PREAMBLE EXTENSION ADDED BY LATEXDIFF

\begin{document}
\vspace{1 in}

\begin{center}
\textbf{PRESIDENTIAL ADDRESS}\\
\vspace{.25in}
55th Annual Meeting of Western Regional Science Association\\
San Diego, California
\end{center}

\vfill
\textsuperscript{1} \small GeoDa Center for Geospatial Computation and
Analysis, School of Geographical Sciences and Urban Planning, Arizona
State University, Tempe, AZ. email: srey@asu.edu

\textsuperscript{2} \small I have benefited from the insightful comments
of \DIFdelbegin \DIFdel{Eric Heikkila who served as discussant for this address}\DIFdelend \DIFaddbegin \DIFadd{Daniel Arribas-Bel, Janet Franklin, Eric Heikkila, Randall Jackson,
Janet Kohlhase, Julia Koschinsky, and Terry Rey}\DIFaddend . Any remaining errors
are my responsibility. \newpage

\section{Introduction}\label{introduction}

The title of this talk could have multiple meanings. If one were to use
\emph{Open} as an adjective, what would follow would be a discussion of
the many ways in which our world of regional science has embraced open
science, an important offspring of the open source revolution. An
alternative meaning would have \emph{Open} as a verb in which case the
talk that follows would be a call to arms for regional scientists to
engage with open science and open source.

As should be clear in what follows, \DIFdelbegin \DIFdel{my view is }\DIFdelend \DIFaddbegin \DIFadd{I hold }\DIFaddend the second interpretation \DIFdelbegin \DIFdel{is the correctone. My reasons for this are two-fold}\DIFdelend \DIFaddbegin \DIFadd{to
be correct, and this for two reasons}\DIFaddend . First, I have personally seen the
impact that open source has had on scientific software development,
primarily through my experiences with the packages STARS (Rey and
Janikas 2006) and PySAL (Rey and Anselin 2010). While software was my
point of entry into this open world, the impacts have spread beyond
better software development models to influence many aspects of my
\DIFdelbegin \DIFdel{day to day }\DIFdelend \DIFaddbegin \DIFadd{day-to-day }\DIFaddend existence as an academic regional scientist. These impacts,
which I outline in more detail below, have been profoundly positive.
Second, I feel that we, the regional science community, have been
largely disengaged from developments in open science. I am excited about
the possibilities that a deeper engagement could \DIFdelbegin \DIFdel{have on }\DIFdelend \DIFaddbegin \DIFadd{offer to }\DIFaddend the future of
regional science and want to do all that I can to \DIFdelbegin \DIFdel{deepen and }\DIFdelend hasten its
development.

To do so I will contrast two worlds of science. The first, and the one
we regional scientists currently find ourselves embedded within, is what
I will call \emph{captured science}. While this is our status quo, it is
not generally what holds everywhere in the broader scientific community
where a second and new type of science is operative. This is what I will
call \emph{open science}. My purpose in this talk is to situate regional
science within the paradigm of open science by arguing that our future
should be linked to open data, open modeling, open software, open
collaboration, and open publication.

I will begin by outlining what exactly I mean by captured science,
highlighting its key characteristics and pointing to some of the
challenges and constraints it poses. Some of these issues may be
familiar to many of us\DIFaddbegin \DIFadd{, }\DIFaddend but I think we, as a community, have not had a
healthy open discussion about their implications. I \DIFdelbegin \DIFdel{next }\DIFdelend \DIFaddbegin \DIFadd{then }\DIFaddend move on to
consider the alternative model of open science, what its components are,
and how regional science might benefit from migrating to this model and
away from the current status quo. I close the talk with some thoughts
about the likely future for open regional science.

\section{Science: Riches and
Challenges}\label{science-riches-and-challenges}

I, and I suspect many of you, take the \emph{science} part of regional
science seriously. We see science as one of the best constructions of
the human kind, running close to beer. We hold it in such high regard as
science has delivered on the noble goals of:

\begin{itemize}
\itemsep1pt\parskip0pt\parsep0pt
\item
  uncovering new knowledge
\item
  deepening our understanding of the world
\item
  improving our realities for the betterment of human kind
\end{itemize}

Many historians point to Descartes's \emph{Discourse on Methods} written
in 1637 as the birth of science. From the beginning the concept of
reproducibility was at the core of science. As Schroeder (2013) notes\DIFaddbegin \DIFadd{,
}\DIFaddend the motto of the Royal Society founded in 1671 was

\begin{quote}
Nullius in Verba\footnote{``Take nobody's word for it.''}
\end{quote}

A key development in the advancement of early science was the decision
by the Royal Society to publish results in the form of letters as a way
to hasten dissemination relative to book publication. Both the format
and process of scientific publishing had self-correction baked into them
from the very beginning. Indeed, the use of three referees as a model
for the review process dates back to this period. At the risk of a gross
simplification, this model has been successful on the whole as the
amount of knowledge produced by the scientific process has grown
enormously. One consequence of this success is that the age of the
generalist has long been eclipsed and it is simply impossible for any
one scientist to be conversant in the ocean of specializations \DIFaddbegin \DIFadd{in which
}\DIFaddend we now swim\DIFdelbegin \DIFdel{in}\DIFdelend .

As a metaphor, consider a map of scientific domains on a sphere where
each domain is represented as a simple polygon. The polygons would
likely be \DIFdelbegin \DIFdel{non-uniform}\DIFdelend \DIFaddbegin \DIFadd{nonuniform}\DIFaddend , but they would represent an exhaustive and
mutually exclusive partitioning of the body of scientific knowledge.
\DIFdelbegin \DIFdel{Now
take }\DIFdelend \DIFaddbegin \DIFadd{Take }\DIFaddend the centroid of each domain polygon as a point representing the
core of that domain. We can \DIFdelbegin \DIFdel{now }\DIFdelend measure the distance between each of the
domains using\DIFdelbegin \DIFdel{say }\DIFdelend \DIFaddbegin \DIFadd{, say, }\DIFaddend great circle distance. We could adopt a spatial
interaction model that incorporates this distance to measure the level
of interaction between two domains $i$ and $j$ at time $t$. We adopt a
time index to reflect the evolution of the associated parameters and
distance metrics.

The impact of increasing distance on inter-core interactions as
knowledge domains expand will of course also be shaped by the evolution
of $\alpha(t)$, our distance decay exponent. Generally speaking, this
will reflect the costs of overcoming the distance between different
bodies of knowledge. My thesis is that, in general terms, we have seen
changes \DIFdelbegin \DIFdel{in }\DIFdelend \DIFaddbegin \DIFadd{from a world in which }\DIFaddend the institutions of science \DIFdelbegin \DIFdel{which }\DIFdelend \DIFaddbegin \DIFadd{that
}\DIFaddend originally worked to dampen this effect to \DIFdelbegin \DIFdel{a world }\DIFdelend \DIFaddbegin \DIFadd{one }\DIFaddend where increasing access
costs have been exacerbating this effect, and this transformation has
everything to do with a walling off of science to capture economic
rents.

\subsection{Captured Science}\label{captured-science}

There are multiple channels through which science has become
increasingly commercialized since the heady days of Descartes. Perhaps
the most widely debated of these is the commercialization of scientific
publishing. The costs of scientific publication \DIFdelbegin \DIFdel{has }\DIFdelend \DIFaddbegin \DIFadd{have }\DIFaddend attracted much
criticism from members of the academy. A well known lament is that the
scientific community carries out the peer review process largely gratis
yet the commercial journal publishers charge billions of dollars per
year for access to the final scientific product.

Although the true \DIFdelbegin \DIFdel{costs }\DIFdelend \DIFaddbegin \DIFadd{cost }\DIFaddend of access to journals is often obscured by
non-disclosure agreements that university libraries must sign to access
journals, some estimates put the annual revenue of the
science-publishing industry at \$9.4 billion in 2011 with an average
revenue per article of roughly \$5,000, average per article costs around
\$3,500-4,000, and profit margins at 20-30\% ({Van Noorden} 2013). These
are Apple-like profit margins\DIFdelbegin \DIFdel{mind you}\DIFdelend . Publishers own the copyright of the paper
and have enclosed the theory within the confines of the article.
Software and data are generally not to be found, even in cases where the
paper access costs can be borne. Captured science is the result.

Commercial publishers counter these claims by pointing to the role that
journal branding can play through the peer review process. Rejection by
high prestige journals serves to filter manuscripts to the most
appropriate outlets and lowers the search costs for future researchers.

\subsection{Open Access}\label{open-access}

A recent development in scientific publishing has been the move to the
so called open access publication model. Although it can take on
different forms, the basic premise is that once a scientific paper has
been published, and requisite fees paid, it is made freely available to
interested readers. The model has gained much momentum\DIFdelbegin \DIFdel{, }\DIFdelend \DIFaddbegin \DIFadd{; }\DIFaddend as of 2011
estimates are that some 50\% of all scientific papers published are in
some form of open access \DIFaddbegin \DIFadd{(Van Noorden 2013). The leading regional
science journals have all adopted some form of open access, although
there is variation in the fees charged to the authors: Annals of
Regional Science (}\$\DIFadd{3000), International Regional Science Review
(}\$\DIFadd{1500), Journal of Regional Science (}\$\DIFadd{3000), Papers in Regional
Science (}\$\DIFadd{3000), Regional Science and Urban Economics (}\$\DIFadd{1800)}\DIFaddend .
\DIFdelbegin \DIFdel{It is useful to take a look at the pressures
that have driven its evolution.
}\DIFdelend 

\DIFdelbegin \DIFdel{Three general
forces have been responsible for open accesscoming into
being}\DIFdelend \DIFaddbegin \DIFadd{Given its rise in popularity, it is important to examine the general
forces behind open access}\DIFaddend . First, and most prominently, is the
widespread disenchantment by the academic community with the traditional
commercial publishing model. This is reflected in such documents as
\emph{The Cost of Knowledge}, signed by some 14,000 scientists who
pledged to no longer participate in commercial scientific publishing,
from refusing to submit their articles to journals under the traditional
model, to declining referee requests or serving in any aspect of the
editorial process. This was a key component of the boycott of Elsevier.

A second pressure reflects movements by governments to begin requiring
that research supported by federal funding make the findings and data
publicly available. This is reflected in the \DIFdelbegin \DIFdel{NSF }\DIFdelend \DIFaddbegin \DIFadd{US National Science
Foundation }\DIFaddend requiring a data management component in all research
proposals with accessibility as \DIFaddbegin \DIFadd{a }\DIFaddend central concern.

The third pressure flowing from academia is reflected in the University
of California's (UC) recent adoption of its Open Access Policy
(University of California 2013) in which faculty grant a license to UC
prior to any contractual agreement with publishers. This permits the UC
to archive the research in its eScholarship system\DIFaddbegin \DIFadd{, }\DIFaddend thus providing
access to the public at no charge.

The open access model has enormous potential for fueling wider
dissemination of and access to scientific research, and it is often held
in high regard by the scholarly community for this very reason. However,
in large part the open access movement can be viewed as a reaction by
commercial publishers to \DIFdelbegin \DIFdel{respond to }\DIFdelend the three pressures mentioned above. Moreover,
the model \DIFdelbegin \DIFdel{has not been }\DIFdelend \DIFaddbegin \DIFadd{is not }\DIFaddend without its problems, some of which are threatening the
basic integrity of scientific publishing.

There have been a number of high profile cases where the poor quality
control of open access journals have been exposed. Particularly striking
is the example outlined in Bohannon (2013) where a single author
submitted 304 versions of the same spoofed article to open access
journals. The key theme in these papers addressed the role of an extract
from lichen, and its anti-cancer properties. Author names for each
version of the paper were randomly created, and the dimensions of the
arguments surrounding combinations of different types of molecules,
species, and cancer cells were exploited to provide some differentiation
of the alternative versions of the paper.

Several profound results stem from this experience. First, more than
half of the papers were accepted. Second, both the key anti-cancer
agents and the author of these manuscripts did not exist. Thirdly, among
the 147 journals that accepted the article were those owned by Elsevier
and Sage, as well as prestigious academic institutions such as Kobe
University. Combined, these results raise doubts about the legitimacy of
the open access journal as a repository for the state of scientific
knowledge as, in this case, acceptance of patently false findings was
far from the exception.\footnote{Questions of selectivity bias in
  Bohannon (2013) have been raised since the sample included only open
  access journals and not journals published under the traditional
  model.}

In addition to the problem of quality control in open access journals,
other major challenges that arise in captured science are the roadblocks
to reproducibility and the weakening of science's self-correction
mechanism. Prominent cases of plagiarism, data cooking, and fraud are to
be found in the social sciences. Frank Fischer, a political scientist at
Rutgers, was accused by a graduate student and Alan Sokal\footnote{The
  namesake of the \emph{Sokal affair} in which the author submitted a
  completely nonsensical manuscript entitled ``Transgressing the
  Boundaries: Towards a Transformative Hermeneutics of Quantum Gravity''
  to the journal \emph{Social Text}. After the paper was published Sokal
  revealed the hoax.} of plagiarism. Similar to the case of Doris Kearns
Goodwin, the accused claimed that it was a simply sloppiness on their
part rather than outright plagiarism.\footnote{For a running commentary
  of plagiarism in the social sciences see the blog of Andrew Gelman at
  http://andrewgelman.com.}

Closer to home for regional scientists is the \DIFdelbegin \DIFdel{Reinhart and Rogoff paper
which purported }\DIFdelend \DIFaddbegin \DIFadd{Rogoff and Reinhart (2010)
paper purporting }\DIFaddend that debt-to-G.D.P. ratios above 90\% dampened growth.
The paper was widely cited in policy circles in support of austerity
programs. However, examination of the study by others revealed a mixture
of spreadsheet errors, omission of available data, favorable weighting
and transcription \DIFaddbegin \DIFadd{errors}\DIFaddend . Follow-up studies have shown the magnitude of
the effect is reduced when these changes are incorporated, but not the
sign (Herndon, Ash, and Pollin 2013). One ray of shining light is that
the heroes of these stories often tend to be graduate students who
uncovered the fraud or questionable practices. They were able to do so
because they had access to the data and methods underlying the original
studies in question.

\subsection{Data Hoarding}\label{data-hoarding}

\DIFdelbegin \DIFdel{In order for }\DIFdelend \DIFaddbegin \DIFadd{For }\DIFaddend science's error-correction mechanisms to kick in\DIFdelbegin \DIFdel{a
necessary condition will be that }\DIFdelend \DIFaddbegin \DIFadd{, }\DIFaddend data underlying
research projects \DIFaddbegin \DIFadd{must }\DIFaddend be made accessible to the wider research
community. Unfortunately, current institutional constraints and
individual practices are standing in the way of realizing this.

\DIFdelbegin \subsection{\DIFdel{Privacy Concerns}}%DIFAUXCMD
\addtocounter{subsection}{-1}%DIFAUXCMD
\DIFdelend \DIFaddbegin \subsubsection{\DIFadd{Privacy Concerns}}\DIFaddend \label{privacy-concerns}

There are important concerns regarding the protection of personal
information on the one hand, and the rich set of empirical analyses that
micro data support on the other. A number of strategies have been
explored to strike this balance. Anonymization of public records
\DIFdelbegin \DIFdel{attempt
}\DIFdelend \DIFaddbegin \DIFadd{attempts }\DIFaddend to minimize the risk of revealing information about
individuals. However, this has its \DIFdelbegin \DIFdel{limit }\DIFdelend \DIFaddbegin \DIFadd{limits }\DIFaddend as a number of high profile
failures have demonstrated.

One example \DIFdelbegin \DIFdel{being }\DIFdelend \DIFaddbegin \DIFadd{was }\DIFaddend the case of the State of Massachusetts Group Insurance
Commission (GIC)\DIFaddbegin \DIFadd{, the agency }\DIFaddend responsible for purchasing health insurance
for state employees (Sweeney 2005). As part of that effort GIC
anonymized data by \DIFdelbegin \DIFdel{removal of }\DIFdelend \DIFaddbegin \DIFadd{removing }\DIFaddend names, addresses, and Social Security
numbers before releasing to researchers. What remained in the released
data were \DIFaddbegin \DIFadd{the }\DIFaddend ZIP code, birth date and gender of each person along with
diagnoses and prescription information. \DIFdelbegin \DIFdel{A }\DIFdelend \DIFaddbegin \DIFadd{For a mere }\$\DIFadd{20 a }\DIFaddend researcher was
able to purchase a voter registration list for Cambridge \DIFdelbegin \DIFdel{for }%DIFDELCMD < \$%%%
\DIFdel{20 }\DIFdelend that contained
the name, address, ZIP code, birth date and gender of each voter.
Linking this with the GIC data, the researcher was able to determine
that only six people had the same birth date, of these three were men,
and only one of these had the same ZIP code. That person was William
Weld, the state's governor.

\DIFdelbegin \subsection{\DIFdel{Data Silos}}%DIFAUXCMD
\addtocounter{subsection}{-1}%DIFAUXCMD
\DIFdelend \DIFaddbegin \subsubsection{\DIFadd{Data Silos}}\DIFaddend \label{data-silos}

Clearly there are problems with anonymization procedures\DIFaddbegin \DIFadd{, }\DIFaddend and a very
active research agenda is developing around privatization (Reiter 2012).
An alternative is the use of safe havens as secure sites for data
containing sensitive \DIFdelbegin \DIFdel{person information}\DIFdelend \DIFaddbegin \DIFadd{personal information, }\DIFaddend with access being granted to
authorized researchers. In the US, Census Research Centers (CRC) play
this role and offer remarkable opportunities for regional scientists to
have access to micro data subject to a number of restrictions designed
to ensure confidentiality.

Although the CRCs are successful in protecting privacy concerns\DIFdelbegin \DIFdel{I think
}\DIFdelend \DIFaddbegin \DIFadd{, }\DIFaddend it
important to keep in mind that they also place limits on the network
effect and science's self-correcting mechanism. This is because
replication of studies that come out of CRC research is difficult\DIFaddbegin \DIFadd{, }\DIFaddend as
any researcher seeking to do so requires access to the same data used in
the original study. That access is only granted by the CRC which faces
difficult choices in determining what proposals get approved for access\DIFdelbegin \DIFdel{,
and given }\DIFdelend \DIFaddbegin \DIFadd{.
Given }\DIFaddend the choice between proposals for new novel studies versus studies
that seek to replicate previous studies, it is conceivable that the
former may be viewed more favorably.\DIFaddbegin \footnote{\DIFadd{CRC also requires that
  research enhance the data used in the approved research projects, and
  this also likely works against proposals that seek to replicate
  previously funded studies.}}
\DIFaddend 

Data hoarding is not limited to institutions as the problem can be found
at the individual level as well. Our existing tenure and reward systems
stress the number of publications produced, and for researchers who have
invested time and resources in constructing or acquiring unique data
sets it is rational to seek a return on that investment by maximizing
their exclusive use of the data. However, the individual scholar model
doesn't really scale well\DIFdelbegin \DIFdel{and }\DIFdelend \DIFaddbegin \DIFadd{, and one must consider }\DIFaddend the opportunity costs
from the scientist restricting access to the data by the wider
scientific community\DIFdelbegin \DIFdel{have to
be considered}\DIFdelend . After all, if the data \DIFdelbegin \DIFdel{is really }\DIFdelend \DIFaddbegin \DIFadd{are really so }\DIFaddend wonderful,
just \DIFdelbegin \DIFdel{image
}\DIFdelend \DIFaddbegin \DIFadd{imagine }\DIFaddend what might flow from releasing it to the field\DIFdelbegin \DIFdel{.
}\DIFdelend \DIFaddbegin \DIFadd{!
}\DIFaddend 

This \DIFdelbegin \DIFdel{doesn't }\DIFdelend \DIFaddbegin \DIFadd{does not }\DIFaddend have to be a zero sum game where the private gains are
sacrificed for social goods - we can tweak the reward structure and
attribution norms to make data provision a first class contribution to
the scientific process. In other words, rather than the single scholar
producing \DIFdelbegin \DIFdel{say }\DIFdelend a series of 5 papers with the exclusive use of the data, she
\DIFdelbegin \DIFdel{produces }\DIFdelend \DIFaddbegin \DIFadd{could produce }\DIFaddend one article and releases the data. This in turn empowers a
larger group of scholars to generate vastly more than 5 papers using the
same data, with each of these papers \DIFdelbegin \DIFdel{siting }\DIFdelend \DIFaddbegin \DIFadd{citing }\DIFaddend the single paper produced by
the data generating/contributing scholar. The citations to her original
article would grow exponentially in this world rather than linearly in
the hoarding model.

\subsection{Modeling Islands}\label{modeling-islands}

Since the early days of regional science\DIFaddbegin \DIFadd{, }\DIFaddend Walter Isard envisioned that
integration would be a hallmark of our discipline (Hewings, Nazara, and
Dridi 2004). It is sobering to contrast that grand vision with today's
state of our modeling science. To a very real extent, integration of
different modeling efforts has fallen far short of this vision. Rather
than a rich ecosystem of interconnected modeling components\DIFaddbegin \DIFadd{, }\DIFaddend the silo
business model appears to have become dominant. In part this reflects
the economics of the regional modeling business where the development,
continued enhancement, and support of modeling frameworks requires
stable and \DIFdelbegin \DIFdel{constant }\DIFdelend \DIFaddbegin \DIFadd{continual }\DIFaddend financial support.

At the same time, we as a community of regional modelers have paid scant
attention to model interoperability. A search of the leading proprietary
regional models (e.g., REMI, IMPLAN) failed to turn up any references to
application programing interfaces (API) \DIFdelbegin \DIFdel{which }\DIFdelend \DIFaddbegin \DIFadd{that }\DIFaddend could be used to couple
different modeling frameworks together. The finger should not just be
pointed at proprietary modeling systems \DIFdelbegin \DIFdel{, }\DIFdelend as the academic community has
also largely ignored interoperability concerns. Indeed\DIFaddbegin \DIFadd{, developers of
}\DIFaddend alternative modeling frameworks are often viewed as competitors rather
than as potential collaborators.

The lack of interoperability has hindered progress in the area of
integrated modeling as much of the research effort has focused on the
challenges of fusing existing modeling frameworks using different
integration strategies (Rey 2000). If model designers had paid more
attention to interoperability, modularity and basic object oriented
practices (Jackson 1994)\DIFaddbegin \DIFadd{, }\DIFaddend less of the research effort would have been
spent on refactoring integration strategies and more on enhancing and
applying integrated models to pressing regional economic issues. This
lack of interoperability is particularly worrisome\DIFaddbegin \DIFadd{, }\DIFaddend given the growing
recognition of the importance of research on coupled \DIFdelbegin \DIFdel{natural-human
}\DIFdelend \DIFaddbegin \DIFadd{natural and human
}\DIFaddend systems and the need for integrated analytical frameworks to support
inquiry.

\section{Open Science}\label{open-science}

While the previous sections \DIFaddbegin \DIFadd{of this address }\DIFaddend have painted a less than
rosy portrait of the current state of science, the brush is not intended
to be overly broad. At the same time, I am also optimistic that rich
opportunities lie before us\DIFaddbegin \DIFadd{, }\DIFaddend and if we grasp them we can affect a
stronger, more open science going forward. Here I outline what I see as
the key pillars of open science and the roles that regional scientists
might play in their realization.

\subsection{Open Source}\label{open-source}

Much attention has been \DIFdelbegin \DIFdel{given }\DIFdelend \DIFaddbegin \DIFadd{paid }\DIFaddend to the open source revolution and its
impacts on many aspects of the modern world (Rey 2009). Remarkably,
these widespread and deep impacts had their origins in the seemingly
obscure question of how software teams organized themselves. As
chronicled by Raymond (1999), the traditional model adopted by
proprietary software houses was to bring together wizards or high
priests working in small groups isolated in towers walled off from users
and markets. The dominant model since the early days of the software
industry, Raymond contrasted this cathedral model with a new upstart
represented by early work on the Linux kernel. Here thousands of
seemingly unorganized and decentralized developers were working to build
the kernel. The chaotic nature of this form of organization struck
Raymond as much more of a bazaar than a cathedral.

That the bazaar model resulted in software that became critical
infrastructure underlying many of the internet services we rely on today
seemed \DIFdelbegin \DIFdel{like an unlikely outcome}\DIFdelend \DIFaddbegin \DIFadd{surprising}\DIFaddend . Its impacts on science are less well recognized, but
we don't have to look far to see concrete evidence. For example, the
beauty contest that is the annual ranking of the worlds fastest super
computers is massively dominated by Linux as the Top500 supercomputer
list reported that 476 of the 500 fastest machines ran Linux
(Vaughan-Nichols 2013).

Perhaps more profound, \DIFdelbegin \DIFdel{but }\DIFdelend \DIFaddbegin \DIFadd{though }\DIFaddend subtle, are the soft innovations that open
source may bring to the practice of science. For science to be truly
open, two components need to be operative. \emph{Open data} constitutes
available, intelligible, accessible and usable data. \emph{Open access}
to scientific publications and knowledge allows the realization of the
building on shoulders of giants.

\subsection{Open Data}\label{open-data}

The increasing availability of open data is playing a pivotal role in
the evolution of the so called \emph{fourth paradigm of science}. The
classic pairing of experiment and theory (first and second paradigms)
were married to the third paradigm of large-scale computational
simulation in the mid-20th century. In this triad, data \DIFdelbegin \DIFdel{has }\DIFdelend \DIFaddbegin \DIFadd{have }\DIFaddend provided
observations about phenomena and \DIFdelbegin \DIFdel{was }\DIFdelend either collected to test particular
theories or generated as output in process based simulations \DIFdelbegin \DIFdel{about }\DIFdelend \DIFaddbegin \DIFadd{of }\DIFaddend those
phenomena. In the fourth paradigm, data \DIFdelbegin \DIFdel{takes }\DIFdelend \DIFaddbegin \DIFadd{assumes }\DIFaddend a more leading role\DIFdelbegin \DIFdel{in
that application }\DIFdelend \DIFaddbegin \DIFadd{.
Applications }\DIFaddend of exploratory and data mining technologies to massive and
heterogeneous datasets are increasingly being used to generate, rather
than test, new hypotheses. Indeed the central role of data in this
context is reflected in an alternative name for this fourth paradigm:
\emph{data-intensive science} (Tolle, Tansley, and Hey 2011).

Regional science has been slow to engage with this new paradigm, and at
times has been hostile to the exploratory nature of much of this work. I
think that hostility reflects traditional paradigms dominating economics
\DIFdelbegin \DIFdel{where theory informed }\DIFdelend \DIFaddbegin \DIFadd{in which theory-informed }\DIFaddend specifications are viewed in high regard, while
exploratory data analysis is often dismissed as measurement without
theory. Yet, in an increasingly interdependent and complex world, one
which is also generating unprecedented and overwhelming amounts of
empirical data, the gap between what extant theory can shed light upon
and what remains unknown will only widen. If we don't change our ways,
we \DIFdelbegin \DIFdel{, regional science, will risk }\DIFdelend \DIFaddbegin \DIFadd{regional scientists will incite }\DIFaddend the criticism of practicing theory
without measurement. In a world of big data, over-reliance on theory
will limit what we can contribute to science.

There are positive developments\DIFaddbegin \DIFadd{, }\DIFaddend however. The rise of exploratory
spatial data analysis, geocomputation, agent based models,
microsimulation, data mining, and related new computationally based
methods are enabling an expansion of the scope of regional science. Very
often the \DIFdelbegin \DIFdel{outcome }\DIFdelend \DIFaddbegin \DIFadd{results }\DIFaddend of these lines of investigation are new types of
questions arising from newly discovered empirical patterns, and this
stands in stark contrast to the tradition approach of hypotheses based
investigation. These do not, however, have to be mutually exclusive and
I would argue they can be complementary approaches.

\subsection{Open Modeling}\label{open-modeling}

Interoperability and open modeling are \DIFdelbegin \DIFdel{vital to our ability to move
}\DIFdelend \DIFaddbegin \DIFadd{crucial for moving }\DIFaddend regional
science into the high performance computing era. Much of our regional
science modeling toolkit consists of frameworks that were developed and
designed for the single desktop era. \DIFdelbegin \DIFdel{In order to tap into
the power offered by }\DIFdelend \DIFaddbegin \DIFadd{Affording the power of
}\DIFaddend multiprocessing, cluster and grid computing architectures \DIFdelbegin \DIFdel{, }\DIFdelend \DIFaddbegin \DIFadd{will require }\DIFaddend a
significant refactoring of this code \DIFdelbegin \DIFdel{base will be
required }\DIFdelend (Anselin and Rey 2012).

While earlier I remarked that the current state of the science in
regional modeling consists of individual modeling efforts that are
largely isolated from one another, there are some important exceptions
that point the way forward towards supporting a genuine form of model
integration. The UrbanSim project (Waddell 2002) \DIFdelbegin \DIFdel{developed since the mid
1990s }\DIFdelend is fully open source
and released under the GPL. Designed for use in operational planning it
has enjoyed wide application across the U.S., Europe, Asia and Africa.
The open source aspect has fueled both the application and development
of UrbanSim, as researchers seeking to apply the modeling framework in
their own projects are free to do so having full access to the source
code.

A second set of exceptions are spatial analytical libraries the are
freely available to regional scientists doing spatial econometrics and
exploratory spatial data analysis. Although not strictly open source in
a licensing sense, Jim LeSage's Spatial Econometric Toolbox was one of
the first freely available (as in price and full source code), libraries
for spatial econometrics. A full open source stack of spatial data
analysis models has been available through the spdep and related
packages in the R environment. Finally, PySAL is an open source library
for exploratory spatial data analysis and spatial econometrics for the
Python language. The key aspect of these libraries is that they provide
detailed APIs for their components\DIFaddbegin \DIFadd{, }\DIFaddend which allows end users to combine
these components in flexible ways

\subsection{Open Collaboration: Release Early and Release
Often}\label{open-collaboration-release-early-and-release-often}

Data analysis is not easy, and honest mistakes \DIFdelbegin \DIFdel{can be, and are , }\DIFdelend \DIFaddbegin \DIFadd{are }\DIFaddend made. Uncovering
those mistakes is vital to sciences self-correction mechanism. Yet,
uncovering those mistakes is also not easy, especially in our current
publishing system. Publication pressures exacerbate this by leaving
scholars precious little time to fully document the research process
that goes into the final publication of \DIFdelbegin \DIFdel{the }\DIFdelend \DIFaddbegin \DIFadd{a given }\DIFaddend manuscript. The lack of
replication infrastructure is a major impediment to identifying errors
of both nefarious and honest species.

One possible antidote for this problem is to adopt more open forms of
collaboration. These would tap into Linus' Law (Himanen 2001):

\begin{quote}
Given enough eyeballs all bugs are shallow.
\end{quote}

A fascinating example of open collaboration is the Polymath Project
developed by Gowers and Nielsen (2009). Inspired by the open source
practices seen in the Linux project and Wikipedia, Gowers began the
project on his blog with a description of a research problem: namely to
develop a proof of the Hales-Jewett Theorem in the realm of
combinatorics. He also provided links for background materials and rules
of engagement designed to encourage peoples' and collaboration.

The project began on February 1 of 2009, and had its first contribution
7 hours \DIFdelbegin \DIFdel{latter }\DIFdelend \DIFaddbegin \DIFadd{later }\DIFaddend from a UBC mathematician. Comments quickly followed from a
diverse array of individuals spanning the spectrum from an Arizona high
school math teacher to a \DIFdelbegin \DIFdel{winner of a Fields Medal . Shortly after
one month from beginning}\DIFdelend \DIFaddbegin \DIFadd{Fields Medal winner. Within a few weeks of its
inception}\DIFaddend , the project \DIFaddbegin \DIFadd{had }\DIFaddend accumulated over 800 contributions \DIFdelbegin \DIFdel{representing }\DIFdelend \DIFaddbegin \DIFadd{comprising
}\DIFaddend 170,000 words and by early March the collective effort had generated an
elementary proof. This initial success of the Polymath project has been
extended to other math problems, and similar open source collaborative
models have been employed in other fields including biology, physics and
computer science. More broadly, one can point to the popularity of sites
such as Stack Exchange, a community-powered question and answer forum
programmers frequent in seeking help on problems, as evidence that this
approach scales well. These models provide new pathways for pushing the
limits of our problem solving abilities.

\DIFdelbegin \DIFdel{I would like to propose a question to all of us here today: ``What would
be an open problem in regional science that we all could contribute
to?'' Don't answer now, we can discuss it over drinks later at the
reception. I will buy a drink for the person who comes up with the best
problem. Actually I am buying you all a drink tonight so I expect a lot
of good problems.
}%DIFDELCMD < 

%DIFDELCMD < %%%
\DIFdelend \subsection{Open Publishing}\label{open-publishing}

Advances in cyberinfrastructure are having impacts not just on how we
\emph{do} science but \DIFdelbegin \DIFdel{are likely to shape the ways we }\DIFdelend \DIFaddbegin \DIFadd{on how we will }\DIFaddend \emph{report} science. The
traditional vessel of reporting findings has been the regular journal
article, which has served us well but, as pointed out above, is showing
its age. Tapping the possibilities of electronic publication opens up
new ways to explore the scientific literature.

The open publication model also provides an entry point to access the
data and methods that underlie an article. This lowers the barriers to
reproducing reported work by other scientists. It is exciting to
contemplate the impact that this could have on research in regional
science. Take the case of the literature on regional convergence where a
number of meta-analyses have attempted a synthesis of what we know about
the processes of regional growth ({de Groot}, Abreu, and Florax 2005).
These entail an enormous amount of traditional literature review and
careful extraction of estimation results from previous studies, the
latter then \DIFaddbegin \DIFadd{being }\DIFaddend used as inputs in meta-regressions to quantify the
relationship between say speed of convergence and aspects of research
design employed in the individual studies.

But\DIFaddbegin \DIFadd{, }\DIFaddend consider a meta-analysis on steroids where what is available to the
meta-researcher is not just data in the form of the final estimation
results of previous papers, but rather the original data, estimation
code and software, and ancillary materials used to generate the reported
estimation results. That is an entirely different ballgame.

Because of our reliance on hard copy journals, too much attention has
been placed on documents as the only research artifact - data and
software are not part of the scientific corpus. Meta-analysis of
research areas becomes prohibitively expensive\DIFaddbegin \DIFadd{, }\DIFaddend and this severely
constrains \DIFaddbegin \DIFadd{the }\DIFaddend synthesis of knowledge. All of this leads to major
problems of irreproducibility\DIFdelbegin \DIFdel{. Essentially what is being comprised is}\DIFdelend \DIFaddbegin \DIFadd{, since}\DIFaddend :

\begin{quote}
Science's capacity for self-correction comes from its openness to
scrutiny and challenges (Boulton 2012)
\end{quote}

If openness is in question, science is in question.

Complexity in doing science, computational burden, and related
technological developments are challenging the traditional scientific
publishing model. No longer does a regular length paper in a hardbound
journal adequately capture the scientific process underlying a research
effort. As a result, reproducibility is almost invariably \DIFdelbegin \DIFdel{not possible
}\DIFdelend \DIFaddbegin \DIFadd{impossible }\DIFaddend in
the current model. The open science model offers a way out of this
morass.

There are signs that alternative open models for scientific publishing
are beginning to gain traction. \DIFdelbegin \DIFdel{A recent example }\DIFdelend \DIFaddbegin \DIFadd{The European Data Watch Extended Project
is building a publication-related data archive to support replication
and reproducibility within economics. As part of that effort it
maintains a comprehensive listing of data availability policies and
replication policies of major economics journals.}\footnote{\DIFadd{http://www.edawax.de/wp-content/uploads/2012/07/Data}\\\DIFadd{SUBSCRIPTNB}{\DIFadd{P}}\DIFadd{olicies}\\\DIFadd{SUBSCRIPTNB}{\DIFadd{W}}\DIFadd{P2.pdf}}
\DIFadd{Increasingly these journals are requiring, as a condition of acceptance,
that the data underlying the paper be made readily available to any
researcher for the purposes of replication and reproducibility.
Unfortunately, none of the five leading regional science journals
discussed earlier have data availability policies.
}

\DIFadd{A second example of an open model for publication, }\DIFaddend that I had experience
with\DIFdelbegin \DIFdel{was the publication }\DIFdelend \DIFaddbegin \DIFadd{, was the production }\DIFaddend of the conference proceedings for the 2013
Scientific Computing with Python (SciPy2013) conference. The entire
process was produced using open source tools, including GitHub\DIFaddbegin \DIFadd{, }\DIFaddend for file
submission, reviewing and ultimately publishing.

Several aspects of this process are worth noting. The process relied on
technologies that were already familiar to this community of scholars
who use code repositories for collaboration on the development of
scientific software on a regular basis. In addition to the technologies,
the traditional roles of author, reviewer, and editor were mapped into
those of participants in an open source software development model. In
this model developers (i.e., authors) wishing to contribute a new
software feature (i.e., article) to a field, issue a pull request (i.e.,
submission) to the project (i.e., journal). That pull request \DIFdelbegin \DIFdel{(manuscript submission) }\DIFdelend is then
reviewed by the community (editor and reviewers) and bug reports
(referee reports) are submitted. The developer \DIFdelbegin \DIFdel{(author) }\DIFdelend then incorporates the
feedback from the bug report (reviews) into the manuscript and updates
the pull request\DIFdelbegin \DIFdel{(revision
submission)}\DIFdelend .

At the end of the process the project maintainer (editor), has to make a
final decision. The pull request would be merged in the case where the
paper is accepted, or simply closed but not merged if the paper were
rejected. In the former case the paper would appear in the final
published proceedings. In contrast to the traditional publication model\DIFaddbegin \DIFadd{,
}\DIFaddend however, papers that were rejected in the process actually remain in the
repository as the trail of pull requests, bug reports and publication
decisions are available for all to see.

Perhaps more importantly, this design \DIFdelbegin \DIFdel{enhanced }\DIFdelend \DIFaddbegin \DIFadd{enhances }\DIFaddend the collaborative nature
of the enterprise\DIFaddbegin \DIFadd{, }\DIFaddend as the reviewers took on roles of allies in helping
to improve the papers. This stood in stark contrast to the traditional
review process in which papers can be shredded by reviewers. At the same
time, the open source model was highly efficient\DIFaddbegin \DIFadd{, }\DIFaddend as the review process
started with an initial pull request deadline of May \DIFdelbegin \DIFdel{19th }\DIFdelend \DIFaddbegin \DIFadd{19 }\DIFaddend and final
publication of the proceedings just over two months later. The
organization of the review process akin to an open source software
development model tapped into the power of community and is a clear
reflection that science publication is not a solitary endeavor but
rather is done by groups of scholars.

It takes a village to publish an article.

\section{Conclusion}\label{conclusion}

\DIFdelbegin \DIFdel{While my hope is that regional science fully }\DIFdelend \DIFaddbegin \DIFadd{Notwithstanding my sincere hope that regional scientists }\DIFaddend embrace open
science, \DIFdelbegin \DIFdel{my
best guess is that }\DIFdelend \DIFaddbegin \DIFadd{more realistically }\DIFaddend the way forward will \DIFdelbegin \DIFdel{represent a mixture }\DIFdelend \DIFaddbegin \DIFadd{likely be forged by
some combination }\DIFaddend of open and proprietary \DIFdelbegin \DIFdel{regional science}\DIFdelend \DIFaddbegin \DIFadd{endeavors}\DIFaddend . In other domains,
this mixed model has been highly successful ({von Hippel} 2004). This
hybrid model will, however, represent a rebalancing of many components
of regional science as it engages with open science.

Our choice is whether that engagement takes an active or passive form.
If it is passive, I fear we will have missed an opportunity for
reinvention. Most of us here work in academia, which has been recognized
as the second most conservative institution every invented, the first
being classic opera (Sui 2014). In other words, we \DIFdelbegin \DIFdel{can't }\DIFdelend \DIFaddbegin \DIFadd{cannot }\DIFaddend wait for the
institutions to change on their own\DIFdelbegin \DIFdel{, }\DIFdelend \DIFaddbegin \DIFadd{; }\DIFaddend we must take a bottom-up,
grass-roots approach to change those institutions.

Each of us in our own way can be a change agent in this regard. This
address represents my attempt towards a contribution, and I hope that
this starts a discussion.

Finally, I will leave you with a recommendation from his \DIFdelbegin \DIFdel{holiness }\DIFdelend \DIFaddbegin \DIFadd{Holiness }\DIFaddend the
Dalai Lama that captures the spirit of open science:

\begin{quote}
Share your knowledge, it is a way to achieve immortality.
\end{quote}

\DIFdelbegin \DIFdel{Thank you for your attention.
}%DIFDELCMD < 

%DIFDELCMD < %%%
\DIFdelend \section{References}\label{references}

Anselin, Luc, and Sergio J. Rey. 2012. ``Spatial Econometrics in an Age
of CyberGIScience.'' \emph{International Journal of Geographic
Information Science} 26: 2211--2226.

Bohannon, John. 2013. ``Who's Afraid of Peer Review?'' \emph{Science}
342 (6154): 60--65.
doi:\href{http://dx.doi.org/10.1126/science.342.6154.60}{10.1126/science.342.6154.60}.
\url{http://www.sciencemag.org/content/342/6154/60.short}.

Boulton, R. 2012. \emph{Science as an Open Enterprise}. London: Royal
Society.

{de Groot}, H. L. F., Maria Abreu, and Raymond J.G.M. Florax. 2005. ``A
Meta-Analysis of Beta-Convergence: the Legendary 2\%.'' \emph{Journal of
Economic Surveys} 19 (3): 389--420.

Gowers, Timothy, and Michael Nielsen. 2009. ``Massively Collaborative
Mathematics.'' \emph{Nature} 461 (7266): 879--881.

Herndon, Thomas, Michael Ash, and Robert Pollin. 2013. \emph{Does High
Public Debt Consistently Stifle Economic Growth?: A Critique of Reinhart
and Rogoff}. Political Economy Research Institute.

Hewings, Geoffrey JD, Suahasil Nazara, and Chokri Dridi. 2004.
``Channels of Synthesis Forty Years on: integrated Analysis of Spatial
Economic Systems.'' \emph{Journal of Geographical Systems} 6 (1): 7--25.

Himanen, Pekka. 2001. \emph{The Hacker Ethic and the Spirit of the
Information Age}. Secker; Warburg.

Jackson, Randall W. 1994. ``Object-Oriented Modeling in Regional
Science: An Advocacy View.'' \emph{Papers in Regional Science} 73 (4):
347--367.

Raymond, Eric C. 1999. \emph{The Cathedral \& the Bazaar: musings on
Linux and Open Source by an Accidental Revolutionary}. Sebastopol:
O'Reilly.

Reiter, Jerome P. 2012. ``Statistical Approaches to Protecting
Confidentiality for Microdata and Their Effects on the Quality of
Statistical Inferences.'' \emph{Public Opinion Quarterly} 76 (1):
163--181.

Rey, Sergio J. 2000. ``Integrated Regional Econometric+input-Output
Modeling: Issues and Opportunities.'' \emph{Papers in Regional Science}
79: 271--292.

---------. 2009. ``Show Me the Code: Spatial Analysis and Open Source.''
\emph{Journal of Geographical Systems} 11: 191--207.

Rey, Sergio J., and Luc Anselin. 2010. ``PySAL: A Python Library of
Spatial Analytical Methods.'' In \emph{Handbook of Applied Spatial
Analysis}, edited by Manfred M. Fischer and Arthur Getis, 175--193.
Berlin: Springer.

Rey, Sergio J., and Mark V. Janikas. 2006. ``STARS: Space-Time Analysis
of Regional Systems.'' \emph{Geographical Analysis} 38 (1): 67--86.

\DIFaddbegin \DIFadd{Rogoff, Kenneth, and Carmen Reinhart. 2010. ``Growth in a Time of
Debt.'' }\emph{\DIFadd{American Economic Review}} \DIFadd{100 (2): 573--78.
}

\DIFaddend Schroeder, William. 2013. ``The New Scientific Publishers.'' In
\emph{Keynote: Scientific Computing with Python}. Austin.

Sui, Daniel. 2014. ``Opportunities and Impediments for Open GIS.''
\emph{Transactions in GIS} 18 (1): 1--24.

Sweeney, Latanya. 2005. ``Recommendations to Identify and Combat Privacy
Problems in the Commonwealth.''
http://dataprivacylab.org/dataprivacy/talks/Flick-05-10.html\#testimony.

Tolle, Kristin M, D Tansley, and Anthony JG Hey. 2011. ``The Fourth
Paradigm: Data-Intensive Scientific Discovery {[}Point of View{]}.''
\emph{Proceedings of the IEEE} 99 (8): 1334--1337.

University of California. 2013. ``UC Open Access Policy.''
\url{http://osc.universityofcalifornia.edu/open-access-policy/}.

{Van Noorden}, Richard. 2013. ``Open Access: The True Cost of Science
Publishing.'' \emph{Nature} 495 (7442): 426--429.

\DIFaddbegin \DIFadd{Van Noorden, Richard. 2013. ``Half of 2011 Papers Now Free to Read.''
}\emph{\DIFadd{Nature}} \DIFadd{500 (7463): 386--387.
}

\DIFaddend Vaughan-Nichols, Steven J. 2013. ``Linux Continues to Rule
Supercomputers.''
\url{http://www.zdnet.com/linux-continues-to-rule-supercomputers-7000016968/}.

{von Hippel}, Eric. 2004. \emph{Democratizing Innovation}. Cambridge:
MIT Press.

Waddell, Paul. 2002. ``UrbanSim: Modeling Urban Development for Land
Use, Transportation, and Environmental Planning.'' \emph{Journal of the
American Planning Association} 68 (3): 297--314.
\DIFaddbegin 

 \DIFaddend\end{document}
